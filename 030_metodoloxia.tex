\chapter{Metodoloxía}

Nesta sección descríbese a metodoloxía levada a cabo para a realización deste proxecto. Unha metodoloxía e un conxunto de métodos ou prácticas empregadas para a realización dunha tarefa, neste caso o analise, deseño e implementación dunha nova ferramenta CAT para o proxecto GNOME. Escolleuse unha metodoloxía axil de ciclo incremental. Moitas das prácticas que se tomaron son collidas da metodoloxía eXtreme Programming.

\section{eXtreme Programming}

Foi creada por Kent Beck no ano 1999 e trátase dun dos máis destacados métodos de desenvolvemento áxil. Os valores fundamentais desta metodoloxía son:

\paragraph{Comunicación} É o primeiro valor de XP. Segundo esta metodoloxía os problemas nos proxectos poden ser traducidos a alguén que non falou con outra persoa sobre algo importante do proxecto. Esta mala comunicación non sucede por casualidade e é debido frecuentemente as malas prácticas. Para solucionar isto, XP inclúe prácticas nas que é necesario a comunicación para levalas a cabo. Ademais a figura do \emph{coach} sirve para mellorar a comunicación daquelas persoas que non o están facendo ben.

\paragraph{Simplicidade} A simplicidade non é unha tarefa sinxela. XP fai unha aposta e invita ao programador a pensar no que o proxecto necesita hoxe e non no que vai necesitar nun futuro. Para manter esta simplicidade ao longo do tempo é frecuente a refactorización do mesmo para manter dita simplicidade. A simplificación do deseño e da implementación axiliza tanto o desenvolvemento como o mantemento. A simplicidade require \textbf{comunicación} pois canto máis comunicación teñamos por parte do cliente máis sinxelo simple poderemos facer o sistema e canto máis simple sexa o sistema menos comunicación será necesaria para explicar o sistema.

\paragraph{Retroalimentación} A retroalimentación ou feedback é fundamental nesta métodoloxía. Pode actuar en varias escalas de tempo. Obtemos feedback en cuestion de minutos ou días de parte dos test do sistema, das peticións dos clientes ou do director do proxecto. Tamen obtemos feedback o longo dos meses cando o usuario pode analizar as caracteristicas que implementamos. Neste sentido XP aposta por unha posta rápida en produción de forma que teñamos sistemas en produción e en desarrollo de forma paralela. Con isto melloramos o sistema xa que imos obtendo as opinións dos usuarios das decisións que xa tomamos e os erros cometidos non se volven a repetir.

\paragraph{Coraxe} O coraxe é unha parte inherente a metodoloxía. É necesario para tirar o traballo de varios días e volver e empezar debido a cambios nos requisitos ou aparición de fallos estructurais. É necesario para ser persistente ca resolución dun problema, as cousas que non se dan resolto un día en horas podense resolver o día seguinte en cuestión de minutos. O coraxe non é útil sen os tres primeiros valores. Con unha boa comunicación existe a posibilidade de facer experimentos con máis risco. A simplicidade permite o programador coñecer mellor o código e polo tanto ser máis valiente a hora de facer cambios. A retroalimentación axuda a que alguén se sinta máis seguro ao facer un cambio.

\paragraph{Respeto} Por último é necesario respeto. É necesario que os integrantes do equipo se preocupen polo resto de membros e polo que están facendo. Ademais o equipo debe preocuparse polo propio proxecto. Para que XP funcione os programadores debense sentir parte do proxecto e ter un feedback positivo ao respecto.

Estos valores son os que segue eXtreme Programming. Pero son necesarias certas prácticas para que estos valores se convirtan nun hábito para o equipo de desarrollo. Algunhas das prácticas incluídas en XP son as seguintes:

\subsection{O Xogo da Planificación}

\subsection{Entregas Pequenas}
As entregas ou \emph{releases} deben ser o máis pequenas posibles e conter os requerimentos máis valiosos. Aínda así cada release debe er autocontenida e non ter características implementadas a medias solo para facer o ciclo de entregas máis curto.

\subsection{Metáfora}
Cada proxecto feito con XP ten unha metafora. Unha metafora é un simil sinxelo do cl

Esta metafora é util para que os membros do equipo teñan unha visión global do que están facendo. Outras metodoloxías chámanlle a isto \textbf{arquitectura}. O problema con empregar o termino arquitectura é que unha arquitectura non ten necesariamente un sentido de cohesión.

O obxetivo desta metafora é ter unha historia coherente para poder explicarlle o sistema tanto os clientes como aos membros do equipo.

\subsection{Deseño simple}
Acorde co principio de deseño KISS (Keep it simple, stupid).

\subsection{Testing}
Calquera caracteristica incorporada qeu non inclua un test, simplemente non existe. Os programadores incluen test de unidade para as novas funcionalidades e os clientes crean test funcionales de como esperan que o programa funciona. Ambos test forman parte do código do programa. Non é necesario escribir test para cada metodo pero si para cada método que se expoña.

\subsection{Refactorización}
Cando é necesario implementar unha nova caracteristica no programa, os programadores preguntanse se existe unha forma simple de implementala e implementana. Despois analizan o código para ver se existe unha forma de facelo de forma máis simple e que siga executando correctamente todos os tests. Esto chamase refactorizar.

É obvio que traballando desta forma emplease moito máis tempo do necesario para a implementación de cada característica pero desta forma poderemos engadir a seguinte caracteristica nunha cantidade razonable de tempo.

\subsection{Programación por parellas}

\subsection{Pertenza Colectiva}
En XP o código pertence a todo o equipo e se unha persoa ten oportunidade de engadir algo de valor a algún fragmento de código ten que facelo nalgún momento. Desta todo o mundo ten responsabilidade de todo o sistema e aínda que non todo o mundo coñece cada parte de forma igual, todo o mundo coñece algo de cada parte de forma que son capaces de facer modificacións satisfactorias.

Isto contrasta coas prácticas doutras metodoloxías onde o código escrito por unha persoa so pertence a esa persoa e para engadir nova funcionalidade é necerio facer unha petición a dito programador. Esta práctica pode facer máis lento o desarrollo e diminue o factor camión\footnote{\href{http://en.wikipedia.org/wiki/Bus\_factor}{Truck Factor}: O numero de membros dun equipo dentro dun proxecto, que no caso de seren atropellados por un camión, o proxecto non podería completarse.}.

\subsection{Integración Continua}
Os test son executados con cada cambio e solo se todos os test son executados correctamente se suben os cambios ao producto final. É importante executar os test a cada cambio xa que asi saberemos a que se debe o fallo e quen ten que correxilo. Se para implementar unha caracteristica os seus desarrolladores non son capaces de que todos os tests funcionen, probablemente necesiten volver a empezar pois non tiñan os coñecementos necesarios para implemementala. 

\subsection{Semana de 40 horas}
XP establece unha xornada laboral de 8 horas e 5 días a semana. Para esta metodoloxía é importante que os programadores estean frescos e inspirados cada maña e con xornadas largas de traballo dita tarefa é imposible. O descanso é algo fundamental para poder ter boas idea.

\subsection{Cliente no sitio}

\subsection{Estandares de programación}

\section{Metodoloxía seguida}


