%
% Portada.
%

% Nota: Sería más cómodo emplear el comando \maketitle que genera una portada de forma automática, pero
% no incluye toda la información que es necesario incluir en la memoria de un proyecto de fin de carrera
% de la Facultad de Informática de A Coruña.
%

\begin{titlepage}

	\begin{center}

		% Logotipo de la universidad.
		\includegraphics[width=6cm]{./eps/logo_udc.eps}
		\vspace{2cm}

		% Nombre de la facultad, de la universidad y del departamento en que se realiza el PFC.
		{\Large{\textbf{Facultade de Informática da Universidade de A Coruña}}}
		\\
		{\it \large{\textbf{Tecnoloxías da Información e das Comunicacións}}}
		\vspace{1cm}

		% Indicamos el nombre de la titulación oficial que hemos cursado con tanto esfuerzo.
		{\large PROXECTO DE FIN DE CARREIRA\\Enxeñería Informática}
		\vspace{1cm}

		% Título
		\textbf{\Large GNOMECAT, un editor de ficheiros GNU Gettext para o proxecto GNOME}
		\vspace{7cm}
	\end{center}

	\begin{flushright}
		\begin{tabular}{ll}
			% Nombre del alumno.
			\large{\textbf{Alumno:}}	&
			\large{Marcos Chavarría Teijeiro} \\

			% Nombre del director/tutor del proyecto.
			\large{\textbf{Director:}}	&
			\large{Fernando Bellas Permuy} \\

			% Fecha.
			\large{\textbf{Data:}}	&
			\large{15 de xuño de 2015} \\
		\end{tabular}
	\end{flushright}

\end{titlepage}
