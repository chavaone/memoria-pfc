%
% Resumen del proyecto de fin de carrera
%

\section*{Resumen:}

A internacionalización e localización de software son dous aspectos moi importantes nos que traballar en produtos software modernos. Que un programa estea localizado a un idioma que o usuario entenda é fundamental para que este usuario se sinta cómodo empregándoo.

O sistema e internacionalización e localización GNU Gettext é unha das solucións para estos dous problemas máis empregada actualmente. Para facer a localización dos programas os tradutores deben editar unha serie de ficheiros. Para facilitar dita tarefa exiten ferramentas denominadas CAT (\emph{Computer Assisted Translation}).

Neste proxecto pretendese crear unha nova ferramenta para a asistencia a tradución centrada na organización de software libre GNOME. Dita ferramenta editara os ficheiros de GNU Gettext e empregará unha interface construída coas ferramntas que aporta o stack de GNOME.

Para facilitar a participación de futuros desenvolvedores empregarase unha linguaxe de programación cunha sintaxe moi amigable en contraste coa empregada en outros proxectos de GNOME. En concreto empregarase a linguaxe de programación Vala.

Este proxecto foi elaborado durante dous verán consecutivos como parte do programa Google Summer of Code.
