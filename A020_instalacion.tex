\chapter{Instrucións de Compilación e Instalación}

GNOMECAT é Software Libre e o seu código fonte está dispoñible  nun repositorio de GitHub de nome \href{http://github.com/chavaone/gnomecat}{chavaone/gnomecat}.

Para poder probar o aplicativo antes de nada debemos instalar as súas dependencias tanto para compilalo como para executalo. O programa depende das seguintes bibliotecas:

\begin{itemize}
 \item glib-2.0
 \item gtk+-3.0
 \item gtksourceview-3.0
 \item gee-0.8
 \item json-glib-1.0
 \item libpeas-1.0
 \item gettext-po
\end{itemize}

Ademais, para descargar e compilar o programa necesitamos as seguintes ferramentas:

\begin{itemize}
\item vala
\item automake e autoconf
\item gettext e intltool
\item git
\end{itemize}

O Fragmento de código~\ref{lst:yum} amosa o comando que temos que executar para instalar as ferramentas e bibliotecas necesarias no caso de que empreguemos o xestor de paquetes \emph{yum} propio de distribucións como Fedora, Red Hat ou CentOS. Se empregamos outros xestores de paquetes o comando sería similar.

\begin{lstlisting}[label=lst:yum,caption=Comando para instalar utilidades e dependencias]
$> sudo yum install git automake autoconf gettext gettext-libs gettext-devel  glib2 glib2-devel gtk+-devel gtksourceview3 gtksourceview3-devel libgee libgee-devel json-glib json-glib-devel libpeas libpeas-devel intltool vala
\end{lstlisting}

Unha vez instalada as dependencias so temos que descarga, compilar e instalar o programa. Como empregamos o sistema de control de versións git, para descargar o repositiore temos que facer uso do seu comando \lstinline{clone}.

Para compilar a aplicación empregaremos os pasos tipicos para intalar calquera aplicación que emprege Autotools. No Fragmento de código~\ref{lst:git} podemos ver os comandos que temos que exectuar para realizar estas tarefas:

\begin{lstlisting}[label=lst:git,caption=Comando para instalar utilidades e dependencias]
$> git clone https://github.com/chavaone/gnomecat.git
$> cd gnomecat
$> ./autogen.sh
$> ./configure
$> make
$> sudo make install
\end{lstlisting}

