\chapter{Escoller a licenza do programa}

Á hora de escoller unha licenza para o novo programa temos que ter en conta as licenzas das bibliotecas empregadas no noso programa. Na siguente lista podense ver as bibliotecas empregadas para o programa así como a licenza de cada unha delas.

\begin{itemize}
  \item \textbf{GLib} GNU Lesser General Public License v2.1
  \item \textbf{GTK+} GNU Lesser General Public License v2.1
  \item \textbf{LibGee}  GNU Lesser General Public License v2.1
  \item \textbf{LibPeas} GNU Lesser General Public License v2.1
  \item \textbf{GettextPo} GNU General Public License v2
  \item \textbf{GTKSourceView} GNU Lesser General Public License v2.1
  \item \textbf{JSON-GLib} GNU Lesser General Public License v2.1
\end{itemize}

Como podemos ver, excepto a biblioteca GettextPO, que emprega unha licenza GNU GPLv2, o resto de bibliotecas empregadas usan unha licenza GNU LGPL v2.1. A licenza GNU Lesser GPL permite empregar calquera licenza en produtos derivados. Por outra parte a licenza GNU GPL v.2 obriga a empregar a licenza GNU GPL v2 ou versións posteriores da mesma licenza.

No noso caso eliximos a licenza GNU General Public Licence v3 que é unha versión máis actualizada da licenza para programas de GNU.
