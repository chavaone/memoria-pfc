\chapter{Escoller a licencia do programa}

A hora de escoller unha licencia para o novo programa temos que ter en conta as licencias das bibliotecas empregadas no noso programa. Na siguente lista podense ver as bibliotecas empregadas para o programa así como a licencia de cada unha delas.

\begin{itemize}
  \item \textbf{GLib} GNU Lesser General Public License v2.1
  \item \textbf{GTK+} GNU Lesser General Public License v2.1
  \item \textbf{LibGee}  GNU Lesser General Public License v2.1
  \item \textbf{LibPeas} GNU Lesser General Public License v2.1
  \item \textbf{GettextPo} GNU General Public License v2
  \item \textbf{GTKSourceView} GNU Lesser General Public License v2.1
  \item \textbf{JSON-GLib} GNU Lesser General Public License v2.1
\end{itemize}

Como podemos ver excepto a biblioteca GettextPO, que emprega unha licencia GNU GPLv2, o resto de bibliotecas empregadas usan unha licencia GNU LGPL v2.1. A licencia GNU Lesser GPL permite empregar calquera licencia en productos derivados. Por outra parte a licencia GNU GPL v.2 obliga a empregar a licencia GNU GPL v2 ou versións posteriores da mesma licencia.

No noso caso eleximos a licencia GNU General Public Licence v3 que é unha versión máis actualizada da licencia para programas de GNU. Na seguinte sección podemos ver o texto completo desta licencia.
