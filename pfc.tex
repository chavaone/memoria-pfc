% Opciones empleadas:
%
% a4paper -> indica el tama? del papel, en este caso A4.
% 11pt -> tama? de la fuente 11 puntos.
% oneside -> s?o escribimos en una cara del folio.
%
% Otras opciones interesantes:
%
% twoside -> escribimos a doble cara.
% openbib -> para que las referencias bibliogr?icas tengan un salto de l?ea entre cada campo de la referencia.
%
\documentclass[a4paper,11pt,oneside]{book}

% codificaci? latin1 y s?bolos del idioma espa?l (? acentos, ...)
\usepackage[spanish]{babel}
\usepackage[utf8]{inputenc}

% puede que queramos usar el s?bolo del euro.
\usepackage{eurosym}

% El paquete fancybox nos permite crear cajas de diferentes estilos con facilidad.
% http://www.ctan.org/get/macros/latex/contrib/fancybox/fancybox.pdf
% http://www.mackichan.com/index.html?techtalk/487.htm~mainFrame
\usepackage{fancybox}
\usepackage{multicol}
\usepackage{amsmath}

% Para incluir subfiguras.
\usepackage{subfigure}

% PARA incluir gr?icos en JPG => compilar con pdflatex.
\usepackage[pdftex]{graphicx}

% Para incluir gr?icos EPS => compilar con latex.
%\usepackage[dvips]{graphicx}

% Para escribir en color...
%
% ... cuando compilamos con el comando ``latex''
%\usepackage[dvips,usenames]{color}
% ... uando compilamos con el comando ``pdflatex''
 \usepackage[pdftex,usenames,dvipsnames]{color}

% Espaciado y ajuste de m?genes
\usepackage{setspace}
\onehalfspacing
% \doublespacing
\setlength{\textwidth}{15cm}
\setlength{\textheight}{22cm}

% Paquete fancyhdr -> Para modificar la cabecera y pie de p?inas.
% http://tug.ctan.org/tex-archive/macros/latex/contrib/fancyhdr/
\usepackage{fancyhdr}
\pagestyle{fancy}
\fancyhf{}
\fancyhf[HR]{\thepage}
\fancyhf[HL]{\nouppercase\rightmark}

% Package booktabs -> Para mejorar el aspecto de las tablas o cuadros.
% http://www.ctan.org/tex-archive/macros/latex/contrib/booktabs/
\usepackage{booktabs}

\usepackage{hyperref}
\hypersetup{
    colorlinks,
    linkcolor={red!50!black},
    citecolor={blue!50!black},
    urlcolor={blue!80!black}
}


% Package rotating -> Para poder girar las tablas y dibujarlas a lo largo
% del folio en vez de a lo ancho.
\usepackage{rotating}

% Packages multicol y multirow, para manejar tablas de filas y columnas m?ltiples.
\usepackage{multicol}
\usepackage{multirow}

\usepackage{color}
\usepackage{xcolor}
\usepackage{caption}
\DeclareCaptionFont{white}{\color{white}}
\DeclareCaptionFormat{listing}{\colorbox{gray}{\parbox{\textwidth}{#1#2#3}}}
\captionsetup[lstlisting]{format=listing,labelfont=white,textfont=white}

 \usepackage{listings}
 \usepackage{courier}
 \lstset{
         basicstyle=\footnotesize\ttfamily, % Standardschrift
         numberstyle=\tiny,          % Stil der Zeilennummern
         numbersep=5pt,              % Abstand der Nummern zum Text
         tabsize=2,                  % Groesse von Tabs
         extendedchars=true,         %
         breaklines=true,            % Zeilen werden Umgebrochen
         keywordstyle=\textbf,
         frame=b,
         stringstyle=\textit, % Farbe der String
         showspaces=false,           % Leerzeichen anzeigen ?
         showtabs=false,             % Tabs anzeigen ?
         xleftmargin=17pt,
         framexleftmargin=17pt,
         framexrightmargin=5pt,
         framexbottommargin=4pt,
         %backgroundcolor=\color{lightgray},
         showstringspaces=false      % Leerzeichen in Strings anzeigen ?
 }

 \lstset{literate=
  {á}{{\'a}}1 {é}{{\'e}}1 {í}{{\'i}}1 {ó}{{\'o}}1 {ú}{{\'u}}1
  {Á}{{\'A}}1 {É}{{\'E}}1 {Í}{{\'I}}1 {Ó}{{\'O}}1 {Ú}{{\'U}}1
  {à}{{\`a}}1 {è}{{\`e}}1 {ì}{{\`i}}1 {ò}{{\`o}}1 {ù}{{\`u}}1
  {À}{{\`A}}1 {È}{{\'E}}1 {Ì}{{\`I}}1 {Ò}{{\`O}}1 {Ù}{{\`U}}1
  {ä}{{\"a}}1 {ë}{{\"e}}1 {ï}{{\"i}}1 {ö}{{\"o}}1 {ü}{{\"u}}1
  {Ä}{{\"A}}1 {Ë}{{\"E}}1 {Ï}{{\"I}}1 {Ö}{{\"O}}1 {Ü}{{\"U}}1
  {â}{{\^a}}1 {ê}{{\^e}}1 {î}{{\^i}}1 {ô}{{\^o}}1 {û}{{\^u}}1
  {Â}{{\^A}}1 {Ê}{{\^E}}1 {Î}{{\^I}}1 {Ô}{{\^O}}1 {Û}{{\^U}}1
  {œ}{{\oe}}1 {Œ}{{\OE}}1 {æ}{{\ae}}1 {Æ}{{\AE}}1 {ß}{{\ss}}1
  {ç}{{\c c}}1 {Ç}{{\c C}}1 {ø}{{\o}}1 {å}{{\r a}}1 {Å}{{\r A}}1
  {€}{{\EUR}}1 {£}{{\pounds}}1
 }

 \usepackage{pgfgantt}

 \usepackage{enumitem}

% Personalizamos la separaci? entre p?rafos...
\parskip=6pt

% Personalizamos el identado en la primera l?ea del nuevo p?rafo...
\parindent=10pt

% Establecemos el n?mero m?imo de niveles de profundidad en las secciones.
\setcounter{secnumdepth}{3}

% T?ulo
\title{GNOMECAT, un editor de ficheiros GNU Gettext para o proxecto GNOME}
% Autor
\author{Marcos Chavarría Teijeiro}
% Fecha
\date{\today}


\renewcommand\lstlistingname{Fragmento de Código}
\renewcommand\lstlistlistingname{Fragmentos de Código}

\newenvironment{bottompar}{\par\vspace*{\fill}}{\clearpage}

\begin{document}

  % \maketitle sirve para generar autom?ica una portada predefinida, pero para un proyecto fin de carrera
  % de FIC no sirvir? porque no cumple las normas de presentaci?. Podemos hacer dos cosas:
  % 1. Usarla e ignorar las normas (y asumir las consecuencias que pueda tener)
  % 2. Hacernos una portada en LaTeX que cumpla las normas (menos arriesgado)
  %
        \include{pfc_portada}

  % FRONTMATTER: TOC, LOF, LOT y descripci?/organizaci? de la memoria.
        \frontmatter

  % Los proyectos de fin de carrera de FIC han de ir acompa?dos de una serie de documentos adicionales, algunos
  %   de ellos obligatorios (certificado, resumen, lista de palabras clave) y otros opcionales (dedicatoria
  % y agradecimientos).
  %
        \thispagestyle{empty}     % No number page, headings...
        \include{pfc_certificado}
        \thispagestyle{empty}     % No number page, headings...

        \include{pfc_resumen}
        \thispagestyle{empty}     % No number page, headings...

        %
% Palabras clave
%

\section*{Lista de palabras clave:}

GNOME, localización, internacionalización, Computer Assisted Translation


        \thispagestyle{empty}     % No number page, headings...

        \include{pfc_dedicatoria}
        \thispagestyle{empty}     % No number page, headings...

        \include{pfc_agradecimientos}
        \thispagestyle{empty}     % No number page, headings...

        \include{pfc_cc_licence}
        \thispagestyle{empty}     % No number page, headings...


        \tableofcontents
        \listoffigures
        %\listoftables


  % MAINMATTER: El contenido, cap?ulo a cap?ulo, de la memoria del PFC.
        \mainmatter
        %
% Frontmatter - Introducción. Los miembros del tribunal que juzgan los PFC's tienen muchas más memorias que leer, por lo que
%	agradecerán cualquier detalle que permita facilitarles la vida. En este sentido, realizar una pequeña introducción,
%	comentar la organización y estructura de la memoria y resumir brevemente cada capítulo puede ser una buena práctica
%	que permita al lector centrarse fácilmente en la parte que más le interesa.
%

\chapter[Introducción]{Introducción}

Lorem ipsum dolor sit amet, consectetur adipisicing elit, sed do eiusmod
tempor incididunt ut labore et dolore magna aliqua. Ut enim ad minim veniam,
quis nostrud exercitation ullamco laboris nisi ut aliquip ex ea commodo
consequat. Duis aute irure dolor in reprehenderit in voluptate velit esse
cillum dolore eu fugiat nulla pariatur. Excepteur sint occaecat cupidatat non
proident, sunt in culpa qui officia deserunt mollit anim id est laborum.

\section{Internacionalización e Localización de Software}
A internacionalización é o proceso de adaptación dun software para que este poida ser adaptado a varios idiomas e ser usado en diferentes rexións sen modificar a súa enxeñería. A localización de software consiste na adaptación de dito software a unha rexión determinada. Isto aínda que afecta fundamentalmente o idioma, tamén ten outros elementos como as divisas a forma de formatear as datas ou simbolos que nunhas areas teñen un significado e noutras outro.

Trátase dun aspecto moi importante do mundo do software pois se ven unha persoa que seipa inglés, o idioma orixe da maior parte dos programas, a internacionalización dos programas é un probelma de comodidade, para o que non entenda a lingua de Shakespeare, trátase dun problema de usabilidade. Unha persoa que non sexa nativa dixital e que non entenda inglés terá serios problemas para entender calquera software moderno non localizado.

\section{O Proxecto GNOME}
GNOME é ambiente de escritorio, unha infraestructura de desarrollo e unha comunidade de software libre.

Como ambiente de escritorio foi creado polos mexicanos Miguel de Icaza e Federico Mena en 1997 como alternativa a KDE compatible coa licencias GPL\footnote{En aquel momento KDE empregaba unha licencia QPL que aínda que era libre non era compatible GPL.}. Tratase de crear un solución software para todo o mundo poñendo interés en aspectos como a accesibilidade, a internacionalización ou a usabilidade.

Como infraestructura de desarrollo GNOME provee unha gran cantidade de aplicativos e bibliotecas para crear o programas tanto para a plataforma GNOME como para outras plataformas.

\begin{figure}[h!]
    \centering
    \includegraphics[width=\textwidth]{img/guadec_2012.png}
    \caption{GUADEC A Coruña 2012}
    \label{fig:guadec2012}
\end{figure}

Como comunidade reune a gran cantidade de persoas tanto voluntarias como profesionais axudaron e axudan a que o proxecto siga adiante. A comunidade GNOME en Europa reunese nas GUADEC, que é o acronimos de \textbf{G}NOME \textbf{U}sers \textbf{A}nd \textbf{D}evelopers \textbf{C}onference. No ano 2012 este encontro tivo lugar na cidade da Coruña.

\section{Localización e Internacionalización no Proxecto GNOME}
Como xa dixemos un dos aspectos máis importante para o proxecto GNOME é o acercamento as persoas e para lograr isto ponse pleno interés na internacionalización e na localización do ambiente de escritorio e das aplicacións de GNOME. GNOME está traducido a máis de 50 idiomas dos cales TODO teñen máis do 90\% das cadeas traducidas.

O proxecto GNOME emprega o sistema GNU Gettext para internacionalizar e localizar os seus programas e conta con unha plataforma web de nome Damned Lies para xestionar a localización de todo o proxecto. Este programa amosa as estatisticas do estado actual das traducións e axuda a xestionar o ciclo de traballo dos traductores. Permite asignar modulos a traductores, que os traductores suban os fichiros descargados e que se faga unha revisión do traballo do traductor.

Outros proxectos de software como Facebook, Twitter ou, dentro do software libre, Ubuntu contan con plataformas online dende as que se pode facer directamente a tradución. GNOME opta por facer a tradución \emph{offline} para que esta sexa dunha maior calidade.

Para que os traductores poidan traballar de forma comoda é necesario a existencia de ferramentas que lle faciliten o traballo. Estas ferramentas coñecense como CAT, que é o acronimo de Computer Assisted Translation. Neste traballo pretende realizarse unha destas ferramentas centrada no proxecto GNOME.

\section{Motivación}

\section{Estrutura da memoria}


In a dolor sed odio eleifend varius. Nam ullamcorper. Curabitur ut erat vulputate nisi molestie tempus. Sed aliquam rutrum odio. In mollis. Fusce consectetuer lorem nec diam. Sed mollis lacinia purus. Curabitur feugiat hendrerit neque. Quisque auctor laoreet diam. Curabitur sit amet nisi. Fusce velit massa, dignissim quis, bibendum eget, vehicula mattis, leo. Morbi auctor leo sit amet nibh. Lorem ipsum dolor sit amet, consectetuer adipiscing elit. Nullam enim. Pellentesque hendrerit, augue non vulputate semper, sem lorem pharetra nibh, sit amet egestas massa diam ac augue. In dui nulla, egestas nec, pulvinar suscipit, tincidunt ornare, nisi. Duis tristique tortor quis magna. Vestibulum faucibus lorem nec neque. Sed nec nibh. Nunc condimentum. Maecenas neque. Nullam pretium est non risus. Etiam gravida. Maecenas nisl. Fusce pharetra odio in tortor. Integer orci turpis, interdum eget, vulputate sed, tristique a, metus. Duis vitae dui quis lectus pretium aliquam. Praesent quam.

Aliquam sed orci. Cras adipiscing nisl quis pede. Ut rhoncus. Donec viverra laoreet purus. Phasellus nulla. Vivamus eget eros. In mollis aliquam orci. Proin ullamcorper. Nullam sollicitudin vestibulum lorem. Nunc malesuada sagittis augue. Donec tellus velit, dapibus a, aliquam ac, tincidunt id, lectus.


\paragraph*{Capítulo 1.}
Phasellus tempor velit nec velit. Proin vitae dui a sapien commodo blandit. Etiam aliquam, sapien vitae fringilla venenatis, lectus sem accumsan orci, eget blandit orci odio et magna. Quisque malesuada, eros vel tempus eleifend, velit enim porttitor sem, eget consequat nulla neque et sapien. Morbi leo. Sed vestibulum lacus. Fusce ut lacus. Phasellus pellentesque pede eu eros. Duis turpis felis, eleifend ut, semper ac, porta nec, sem. Praesent odio. Sed laoreet mollis purus. Praesent vestibulum, velit ut mollis aliquam, quam lectus varius urna, sed ultricies erat nisl ac tortor. Vivamus tempor mauris sit amet nulla. Integer venenatis. Integer sagittis euismod ante. Suspendisse at elit. Duis eget purus nec pede adipiscing auctor. Proin ac est.

\paragraph*{Capítulo 2.}
Proin condimentum. Maecenas sodales. In ornare nunc a leo. Nam sit amet ligula. Nunc quis urna ac metus imperdiet lobortis. Sed quis ligula. Maecenas blandit pede. Donec lacinia rutrum ligula. Vivamus in metus vel elit pharetra molestie.

\paragraph*{Capítulo 3.}
Nullam ante lorem, placerat et, egestas nec, pellentesque non, sapien. Donec semper, felis id posuere faucibus, nibh ipsum tincidunt quam, et varius ipsum odio ac neque. In tincidunt dignissim diam. Sed lacus lorem, ornare ut, eleifend vel, pellentesque tempus, augue. Duis eu magna. Mauris libero ante, porttitor vel, lobortis a, mollis ac, sem. Nunc at lectus. Integer ac libero a nisl dignissim mollis. Donec velit neque, vestibulum eget, pulvinar vel, malesuada ut, nisi. Praesent congue tempus quam. Cum sociis natoque penatibus et magnis dis parturient montes, nascetur ridiculus mus.

\paragraph*{Capítulo 4.}
Mauris ut odio. Nulla accumsan. Morbi condimentum fermentum purus. Pellentesque habitant morbi tristique senectus et netus et malesuada fames ac turpis egestas. Nunc dignissim, neque eget convallis pretium, diam tortor fringilla lacus, a laoreet nisl metus eu magna. Cras ut lectus. Etiam accumsan feugiat elit.

\paragraph*{Capítulo ...}
Donec a pede. Proin dolor. Ut nunc ligula, tempor id, ornare sit amet, aliquam et, nibh. In mollis iaculis pede. Vivamus gravida orci eu nisl. Sed nibh sem, consequat at, iaculis non, placerat in, ligula. Praesent id nisi. Nunc pellentesque justo non libero. Sed quis est sit amet purus lobortis blandit. Sed arcu justo, rhoncus condimentum, ullamcorper iaculis, viverra et, nisl.

\paragraph*{Capítulo N.}
Fusce luctus gravida leo. Nullam dignissim arcu ac risus hendrerit rhoncus. Aliquam erat volutpat. Ut mollis, mauris non aliquam luctus, nulla sem aliquam tellus, in consequat augue odio in urna.





        \chapter[Estado do arte]{Estado do Arte\label{estado_do_arte}}

Neste capítulo explicarase como funciona a internacionalización e localización con GNU Gettext. Ademais analizaremos as diferentes alternativas que existen no mercado como ferramentas de asistencia a tradución e as características que cada unha incorpora. O final faremos un resumo das características que empregan os programas CAT\footnote{Computer Assisted Translation}.

\section{Internacionalización e localización con GNU Gettext}
Gettext é un sistema para a internacionalización e localización amplamente usado en entornos UNIX. Conta con varías implementacións, sendo a primeira de Sun Microsystems no ano 1990. A implementación máis usada é a que GNU liberou no ano 1995. Pese ser unha solución antiga, é a día de hoxe a mellor que se pode atopar no mercado.

Para internacionalizar un programa con GNU Gettext non empregaremos as cadeas de texto directamente como podería ser no seguinte programa de exemplo:

\begin{lstlisting}[language=C,label=some-code,caption=helloworld.c (Sen Internacionalizar)]
#include <stdio.h>

int
main ()
{
    printf ("Hello World!");
}
\end{lstlisting}

En lugar diso chamaremos a unha función especial que proporciona Gettext de nome \lstinline{gettext()} pero que é máis empregada a través do seu alias \lstinline{_()}. Ademais configuraremos o programa para que colla a tradución do idioma que queiramos. Desta forma o programa anterior quedaría:

\begin{lstlisting}[language=C,label=some-code,caption=helloworld.c]
#include <stdio.h>
#include <locale.h>
#include <libintl.h>

#define _(str) gettext(str)

#define

int
main ()
{
    setlocale (LC_ALL, "");
    bindtextdomain ("helloworld", "/usr/local/share/locale");
    textdomain ("helloworld");

    printf (_("Hello World!"));
}
\end{lstlisting}

A función \lstinline{gettext()} é a encargada de substituír a cadea orixinal pola tradución. Non obstante vemos que debemos configurar algunhas cousas antes de poder chamar á función.

En primeiro lugar debemos establecer a linguaxe que queremos empregar no programa. Para iso usamos a función \lstinline{setlocale()}. O primeiro argumento da función determina que parte do locale actual queremos modificar. Entre outras podemos atopar:

\begin{itemize}
    \item \textbf{LC\_ALL.} Queremos cambiar todo.
    \item \textbf{LC\_ADDRESS.} Queremos cambiar a forma de formatar os enderezos.
    \item \textbf{LC\_MESSAGES.} Os mensaxes do programa.
    \item \textbf{LC\_NUMERIC.} O formatado das cantidades non monetarias
    \item \textbf{LC\_TIME.} O formatado de datas e horas.
\end{itemize}

Por último especificamos o código de idioma ou no caso de empregar a cadea baleira empregamos os valores das variables de entorno.

Os códigos de idioma empregan a normativa ISO 639 polo que son da forma $$language[\_territory][.codeset][@modifier]$$ Por exemplo o código do galego empregando codificación UTF-8 é \lstinline{gl\_ES.UTF-8}.

Ademais debemos indicarlle o programa onde ten que atopar as traducións. Para iso empregamos a función \lstinline{bindtextdomain()} que liga un nome de dominio a unha ruta dentro do sistema e a función \lstinline{textdomain()} que lle indica o programa cal é o nome de dominio que debe empregar. Un dominio é un conxunto de cadeas que se empregan nunha parte determinada dun programa. Cada dominio debe ter un nome de dominio único dentro dun programa.

Con estes parámetros Gettext xa é capaz de atopar as traducións que no caso do programa anterior atoparíanse en \emph{/usr/local/share/locale/gl/LC\_MESSAGES/helloworld.mo}.

Unha vez que internacionalizamos o nosos programa debemos traducir as cadeas. Pero para traducir as cadeas debemos extraelas antes do código fonte. Para iso empregaremos a utilidade \emph{xgettext}. Empregando as opcións adecuadas obtemos o seguinte ficheiro:

\begin{lstlisting}[label=some-code,caption=helloworld.pot]
# SOME DESCRIPTIVE TITLE.
# Copyright (C) YEAR THE PACKAGE'S COPYRIGHT HOLDER
# This file is distributed under the same license as the PACKAGE package.
# FIRST AUTHOR <EMAIL@ADDRESS>, YEAR.
#
#, fuzzy
msgid ""
msgstr ""
"Project-Id-Version: PACKAGE VERSION\n"
"Report-Msgid-Bugs-To: \n"
"POT-Creation-Date: 2014-11-12 19:24+0100\n"
"PO-Revision-Date: YEAR-MO-DA HO:MI+ZONE\n"
"Last-Translator: FULL NAME <EMAIL@ADDRESS>\n"
"Language-Team: LANGUAGE <LL@li.org>\n"
"Language: \n"
"MIME-Version: 1.0\n"
"Content-Type: text/plain; charset=CHARSET\n"
"Content-Transfer-Encoding: 8bit\n"

#: helloworld.c:14
#, c-format
msgid "Hello World!"
msgstr ""
\end{lstlisting}

Os ficheiros Gettext coa estensión \emph{POT} trátanse de plantillas xenéricas para todos os idiomas. Para obter o arquivo especifico para o noso idioma debemos empregar a ferramenta \emph{msginit} coa que obteremos un ficheiro similar a este:

\begin{lstlisting}[label=some-code,caption=helloworld.po (Sen Traducir)]
# Galician translations for HELLOWORLD package.
# Copyright (C) 2014 THE HELLOWORLD COPYRIGHT HOLDER
# This file is distributed under the same license as the ch package.
# Marcos Chavarría Teijeiro <chavarria1991@gmail.com>, 2014.
#
msgid ""
msgstr ""
"Project-Id-Version: ch 01\n"
"Report-Msgid-Bugs-To: \n"
"POT-Creation-Date: 2014-11-12 19:24+0100\n"
"PO-Revision-Date: 2014-11-12 19:54+0100\n"
"Last-Translator: Marcos Chavarría Teijeiro <chavarria1991@gmail.com>\n"
"Language-Team: Galician\n"
"Language: gl_ES\n"
"MIME-Version: 1.0\n"
"Content-Type: text/plain; charset=ISO-8859-1\n"
"Content-Transfer-Encoding: 8bit\n"

#: helloworld.c:14
#, c-format
msgid "Hello World!"
msgstr ""
\end{lstlisting}

Desta forma obtemos un ficheiro Gettext PO que é o ficheiro que temos que editar. Traducindo o arquivo obtemos algo como isto:

\begin{lstlisting}[label=lst:translated_example,caption=helloworld.po (Traducido)]
# Galician translations for HELLOWORLD package.
# Copyright (C) 2014 THE HELLOWORLD COPYRIGHT HOLDER
# This file is distributed under the same license as the HELLOWORLD package.
# Marcos Chavarría Teijeiro <chavarria1991@gmail.com>, 2014.
#
msgid ""
msgstr ""
"Project-Id-Version: HELLOWORLD 1.0\n"
"Report-Msgid-Bugs-To: \n"
"POT-Creation-Date: 2014-11-12 19:24+0100\n"
"PO-Revision-Date: 2014-11-12 19:54+0100\n"
"Last-Translator: Marcos Chavarría Teijeiro <chavarria1991@gmail.com>\n"
"Language-Team: Galician\n"
"Language: gl_ES\n"
"MIME-Version: 1.0\n"
"Content-Type: text/plain; charset=ISO-8859-1\n"
"Content-Transfer-Encoding: 8bit\n"

#: helloworld.c:14
#, c-format
msgid "Hello World!"
msgstr "Ola Mundo!"
\end{lstlisting}

Antes de poder empregar o ficheiro no noso programa temos que compilalo. Para iso empregamos a utilidade msgfmt ca que obtemos o ficheiro helloworld.mo. Se movemos o ficheiro o directorio adecuado (o que especificamos en \lstinline{textdomain()}) o noso programa xa estará localizado.

%http://www.gnu.org/software/libc/manual/html_node/Locating-gettext-catalog.html

\subsection{Ficheiros Gettext PO}
Como xa dixemos antes os ficheiros PO son os ficheiros que temos que editar para localizar o noso programa. Primeiro dicir que se trata ficheiros de texto plano e que polo tanto podemos editar con calquera editor de ficheiros de texto plano. Non obstante, o ideal é empregar algunha ferramenta que nos facilite a tarefa como pode ser unha ferramenta CAT.

As súas principais características son:

\paragraph{Soporte de plurais}
Algo que pode parecer trivial como o soporte de plurais deixa de selo cando consideramos que non todos as linguaxes do mundo empregan dous plurais. A lingua eslovaca, por exemplo, conta con tres formas de plural de forma que o plural faise diferente para 1, 3 e 5 elementos.

Gettext representa a forma de plural de cada linguaxe con unha cadea da seguinte forma: $$nplurals=n; plural=exp;$$ Onde $n$ representa o número de plurais da linguaxe e $exp$ a expresión para calcular cando debemos empregar cada forma. Por exemplo a forma plural do galego representase como $nplurals=2; plural=(n != 1);$. Isto é que temos 2 plurais e que so se emprega a forma singular cando o número de elementos é igual a $1$.

No código fonte para que GetText escolla a tradución adecuada temos que empregar a función \lstinline{ngettext}. Esta función recibe como parámetros a cadea orixinal en singular, a cadea orixinal en plural e o número de elementos. No seguinte fragmento de código temos un exemplo:

\begin{lstlisting}[language=C,caption=Plurais en GetText (Código Fonte).]
[...]
    printf (ngettext ("We have %d car.", "We have %d cars.", n), n);
[...]
\end{lstlisting}

A cadea do ficheiro PO correspondente o código anterior pódese ver no seguinte fragmento de código. Vemos como temos unha entrada \lstinline{msgstr} por cada plural. Desta forma o plural número $0$ corresponde os singular é o plural número $1$ correspondese coa primeira forma do plural.

\begin{lstlisting}[caption=Plurais en GetText (Ficheiro PO).]
#: helloworld.c:19
#, c-format
msgid "We have %d car."
msgid_plural "We have %d cars."
msgstr[0] "Temos %d coche."
msgstr[1] "Temos %d coches."
\end{lstlisting}


\paragraph {Marcado de traducións difusas}
Permítese marcar certas traducións como difusas de forma que o tradutor indica que non estar seguro de que dita tradución sexa correcta. Se marcásemos a tradución \emph{"Hello World!"} como difusa o ficheiro PO tería o seguinte aspecto:

\begin{lstlisting}[label=some-code,caption=Ficheiro POT con comentario.]
[...]
#: helloworld.c:15
#, c-format
#, fuzzy
msgid "Hello World!"
msgstr ""
[...]
\end{lstlisting}

\paragraph {Formato das traducións}
Os ficheiro PO permite indicar se as cadeas a traducir teñen un formato determinado. Por exemplo a cadea do exemplo no Fragmento de Código \ref{lst:translated_example} pódese ver que ten o flag \lstinline{c-format} debido a que é parte dunha sentencia printf e podería levar indicadores de formato da forma \lstinline{%s}.

\paragraph {Cabeceira con metadatos}
Existe unha cadea especial nos documentos Gettext PO. Trátase da cadea baleira que serve para almacenar metadatos do ficheiro. No fragmento de código \ref{lst:translated_example} podemos ver istos metadatos. Algúns dos metadatos existentes son:

\begin{itemize}
    \item \textbf{Project-Id-Version.} Nome único para o proxecto deste arquivo de tradución.
    \item \textbf{Report-Msgid-Bugs-To.} Ligazón onde reportar errores nas cadeas orixinais ou para pedir contexto para facer a tradución.
    \item \textbf{POT-Creation-Date.} Data de creación do ficheiro POT.
    \item \textbf{PO-Revision-Date.} Data da última actualización das traducións.
    \item \textbf{Last-Translator.} Nome e enderezo de correo electronico do último traductor.
    \item \textbf{Language-Team.} Enderezo de correo eléctronico do equipo de traductores.
    \item \textbf{Language.} Linguaxe do ficheiro expresada coa codificación ISO 639.
    \item \textbf{Content-Type.} Tipo MIME do ficheiro, que será sempre \lstinline{text/plain} e codificación dos caracteres.
    \item \textbf{Plural-Forms.} Expresión da forma plural empregada.
\end{itemize}

Ademais destes campos, nos comentarios, gárdanse os nomes de todas as persoas que contribuíron a esta tradución.

\paragraph{Gardado dos orixes das cadeas}
Gettext almacena para cada cadea en que lugares do código aparece esta. O cal pode ser moi interesante para implentar a previsualización das traducións. Por exemplo no Fragmento de Código~\ref{lst:translated_example} vemos como a cadea \emph{"Hello World!"} pode atoparse na liña 14 do ficheiro \lstinline{helloworld.c}.

\paragraph{Comentarios dos programadores}
É unha función moi importante xa que en moitas ocasións nas linguaxes a mesma palabra empregase como verbo ou como nome polo que en ocasións é importante incorporar un contexto para esa tradución. Para facer isto é necesario simplemente poñer un comentario no programa antes de empregar a cadea. Por exemplo no seguinte fragmento de código:

\begin{lstlisting}[language=C,caption=Tradución con comentario.]
[...]
    // Translators: We are just waving the world.
    printf (_("Hello World!"));
[...]
\end{lstlisting}

Estamos engadindo un comentario a cadea \emph{"Hello World!"}. O ficheiro POT resultado tería a forma:

\begin{lstlisting}[label=some-code,caption=Ficheiro POT con comentario.]
[...]
#. Translators: We are just waving the world.
#: helloworld.c:15
#, c-format
msgid "Hello World!"
msgstr ""
[...]
\end{lstlisting}


\paragraph{Comentarios dos tradutores}
A biblioteca permite que os tradutores comenten as cadeas. Engadindo un comentario á cadea \emph{"Hello World!"}, o ficheiro PO tería o seguinte aspecto:

\begin{lstlisting}[caption=Ficheiro PO con comentario.]
[...]
# This is a note from translators.
#: helloworld.c:15
#, c-format
msgid "Hello World!"
msgstr ""
[...]
\end{lstlisting}

\section{Ferramentas CAT do mercado}
\label{sec:ferramentascat}
Nesta sección analizaremos algunhas das ferramentas de asistencia a tradución existentes. Veremos as características que incorporan estes programas así como estudar a súa interface de usuario.

\subsection{GTranslator}
GTranslator é a aplicación oficial do proxecto GNOME para a asistencia a tradución. Este aplicativo so permite a tradución de arquivos GNU Gettext. As característica máis destacables deste programa son a posibilidade de abrir varios ficheiros en diferentes lapelas, soporte de memorias de tradución, perfiles para diferentes tradutores, edición dos comentarios dos ficheiros .po e un sistema de plugins que permite estender a ferramenta.

\begin{figure}[h]
    \centering
    \includegraphics[width=\textwidth]{img/captura_gtranslator.png}
    \caption[Interface de GTranslator]{Interface de GTranslator}
    \label{fig:gtranslator}
\end{figure}

En canto a interface, como podemos ver na Figura~\ref{fig:gtranslator}, a parte máis importante do programa é a a lista de mensaxes. Abaixo desta lista temos un panel onde se pode editar cada mensaxe e a súa dereita a memoria de tradución. A disposición dos elementos desta interface é configurable xa que permite mover e ampliar cada un dos módulos. O programa tamén incorpora atallos de teclado que permiten moverse polo documento e seleccionar cada elemento da memoria de tradución.

Este programa pese a ser o aplicativo oficial de GNOME é moi pouco usado. As razóns disto son a ausencia dunha característica chave que o diferencie doutras ferramentas do mercado, a presencia de fallas importantes que afectan a usabilidade e a ausencia dun mantedor que resolva estes problemas.

\subsection{Lokalize}
Lokalize é o programa oficial para o soporte a tradución en KDE. Foi escrita dende cero empregando a tecnoloxía de KDE Platform 4 e baseándose no código de KBabel. As súas características máis destacables son, o soporte para ficheros GNU Gettext e o formato QT TS entre outros; a xestión de proxectos incorporando unha vista que permite ver un resumo de cada ficheiro dentro do proxecto; uso de memorias de tradución e de glosarios; comprobación da ortografía e vista previa das traducións a partir de scripts feitos polo usuario.

\begin{figure}[h]
    \centering
    \includegraphics[width=\textwidth]{img/captura_lokalize.png}
    \caption{Interface de Lokalize}
    \label{fig:lokalize}
\end{figure}

En canto a interface, o programa permite abrir varios ficheiros cada un na súa lapela. Como se pode ver na Figura~\ref{fig:lokalize}, a vista de tradución está centrada no panel de edición, onde aparecen tanto a cadea orixinal como a cadea a traducir. Na columna da esquerda podemos ver a lista de cadeas onde se amosan os primeiros caracteres da cadea orixinal e da tradución e o estado desta tradución. Ademais tamén se pode ver o contexto da tradución, engadir un novo comentario e ver en que ficheiros estaba dita tradución. Por último tamén podemos ver na parte inferior a memoria de tradución e o glosario.

\subsection{Virtaal}

Virtaal é unha ferramenta CAT creada por Translate House\footnote{Compañía que surxiu a partir dunha comunidade de tradutores de Sudáfrica e que está especializada na creación de ferramentas e bibliotecas para axudar a tradución.} As características máis destacables son a incorporación de suxestión a tradución, comprobación da calidade das tradución e, sobretodo, a capacidade de abrir unha gran variedade de formatos a través da biblioteca Translate Toolkit.

\begin{figure}[h]
    \centering
    \includegraphics[width=\textwidth]{img/captura_virtaal.png}
    \caption{Interface de Virtaal}
    \label{fig:virtaal}
\end{figure}

Como se pode ver na Figura~\ref{fig:virtaal}, a interface é minimalista e moi centrada na tradución. A diferencia dos casos anteriores, trátase dunha interface fixa e que integra a lista das mensaxes coa edición da propia mensaxe. Tamén se indican posibles fallos que poidan ter a tradución, como falta de puntos o final, ausencia de marcadores de formato, etc.

\subsection{OmegaT}
Ferramenta CAT lanzada no ano 2001 e pensada fundamentalmente para tradutores profesionais. Ten soporte para o uso de memoria de tradución, glosario, tradución directa entre outras cousas. Destaca a gran cantidade de formatos que pode empregar.

\begin{figure}[h]
    \centering
    \includegraphics[width=\textwidth]{img/captura_omegat.png}
    \caption{Interface de OmegaT}
    \label{fig:omegat}
\end{figure}

A interface de OmegaT dista bastante do resto de programas. Como amosa a Figura~\ref{fig:omegat} non existe o concepto de lista de mensaxes traducir. O documento amosase como unha sucesión de cadea e se facemos clic encima dunha, permitiranos traducila. Trátase dun programa moi usado para a tradución tanto profesional como amateur.

\subsection{Google Translation Toolkit}
É a ferramenta CAT desenvolvida por Google e lanzada no ano 2008. A diferencia dos aplicativos analizados anteriormente, esta trátase unha solución puramente web. Entre as súas principais características encontrase a posibilidade de facer tradución automática empregando Google Translator, o uso de memorias de tradución compartidas, glosarios, soporte de etiquetas HTML entre outros e atallos de teclado. Ten soporte para varios formatos como ficheiros PO, documentos de Microsoft Word, de LibreOffice ou mesmo artigos da Wikipedia.

\begin{figure}[h]
    \centering
    \includegraphics[width=\textwidth]{img/captura_googletranslationtoolkit.png}
    \caption{Interface de Google Translation Toolkit}
    \label{fig:translatetoolkit}
\end{figure}

Como se pode ver na Figura~\ref{fig:translatetoolkit}, na interface misturase a lista de cadeas co cadro de edición das mesmas. Ademais empregando unha interface semellante a do resto de ferramentas ofimáticas de Google, temos botóns para autocompletar tags e para avanzar a seguinte tradución. Aínda que foi pensado para a tradución colaborativa de documentos de ONGs e artigos da Wikipedia, na actualidade emprégase maioritariamente para a tradución de proxectos comerciais.


\subsection{Transifex}
Trátase dunha plataforma que xurdiu a partir dun proxecto do Google Summer of Code do ano 2007 que pretendía crear unha plataforma online máis amigable que o Damned Lies de GNOME que naquel momento tamén empregaba Fedora unha distribución de GNU/Linux. Trátase dunha solución de pago con plans que van dende os 19 a os 300 dólares. Non obstante os proxectos de código aberto poden usar o servizo de forma gratuíta e dispón de un período de mostra 30 días. As súas principais características son a posibilidade de descargar o documento e volvelo a subir para poder traducilo con outra ferramenta CAT, editor online, memoria de tradución e unha API que permite integralo con outros servizos. Ademais tamén soporta unha gran variedade de formatos entre os que se atopan os ficheiros PO, DTD de Mozilla ou XML entre outros.

\begin{figure}[h]
    \centering
    \includegraphics[width=\textwidth]{img/captura_transifex.png}
    \caption{Interface de Transifex}
    \label{fig:transifex}
\end{figure}

A interface de trasifex, como ser pode ver na Figura~\ref{fig:transifex} separa a lista de cadeas do cadro de edición. No cadro de edición temos a posibilidade de consultar a memoria de tradución ou o glosario. O programa tamén incorpora resaltado de sintaxe.

\subsection{Outras ferramentas}
Existen moitas máis ferramentas CAT no mercado. De feito, segundo unha enquisa \cite{article:2006survey} elaborada polo Imperial College London a cerca de 900 tradutores profesionais de 54 países diferentes, os únicos programas de todos os anteriores que aparecen citados é o OmegaT que conta con un 7\% de usuarios. As ferramentas máis usadas son ferramentas para Microsoft Windows e ferramentas con licencias privativas e usualmente moi caras. Algunhas destas ferramentas son TRADOS, Wordfast, DejaVu SDLX ou STAR Transit. Na Figura~\ref{fig:enquisa2006} podemos ver unha gráfica coas ferramentas máis empregadas segundo este estudio.

\begin{figure}[h]
    \centering
    \includegraphics[width=0.7\textwidth]{img/grafico_uso_cat_enquisa2006.png}
    \caption{Ferramentas CAT máis empregadas segundo \cite{article:2006survey}.}
    \label{fig:enquisa2006}
\end{figure}

Hai que ter tamén en conta que se trata dun estudio bastante antigo polo que algunhas das ferramentas analizadas aínda non existían. Por exemplo a enquisa cita a ferramenta KBabel, que é a ferramenta de KDE na que está baseada Lokalize, pero con menos dun 2\% de usuarios.


\section{Características xenéricas das ferramentas CAT}

Algunhas das características que aparecen de forma recurrente en todas as ferramentas analizadas son as seguintes:

\subsection{Memoria de Tradución}
Unha memoria de tradución é unha base de datos composta de textos orixinais acompañados das súas traducións. Estes textos almacénanse en segmentos onde a separación entre segmentos ven dada por signos de puntuación ou o cambio de parágrafo, sendo esta última forma a máis frecuente.

A principal función dunha memoria de tradución e a extracción de coincidencias totais ou parciais. Os programas que teñen esta característica buscan na base de datos un segmento que coincida de forma exacta ou parcial coa cadea que se está a traducir e mostrase este segmento como suxerencia. Xunto coa suxerencia tamén se amosa o grado de cercanía entre a cadea a traducir e a cadea da memoria de tradución.

Existe un formato estándar de compartición de memorias de tradución de nome Translation Memory eXchange (TMX) e plataformas online que almacenan gran cantidade de cadeas e polo tanto hai máis posibilidade de obter unha mellor coincidencia. O software Amagama creado por Translate House, os creadores de Virtal, é un exemplo de memoria de tradución online.

\subsection{Glosario}
Un glosario é unha base de datos de termos xunto con unha ou varias traducións aceptadas. Diferenciase da memoria de tradución en que so se proporcionan a tradución a termos e non a cadeas completas. De igual forma que no caso das memorias de tradución, existe un formato estándar de para a compartición de glosarios de nome TermBase eXchange (TBX) e plataformas online para almacenar os glosarios.


\subsection{Previsualización}
As funcións de previsualización permítenlle ó tradutor ver como vai quedar a cadea traducida no programa final.

Os programadores e deseñadores fan as interfaces tendo en conta a lingua orixinal e non ningunha das traducións polo que se unha tradución é moito máis longa ca orixinal pode verse mal no programa final.

Para conseguir esta característica pódense empregar varias técnicas:

\begin{itemize}
  \item \textbf{Programa orixinal.} Esta técnica que se pode empregar en calquera cadea consiste en compilar o ficheiro PO e executar o programa final con ese ficheiro PO. Desta poderemos ver como queda a nosa tradución para o usuario final. A desvantaxe deste método consiste en que teremos que saber en que parte do programa se emprega a cadea que queremos previsualizar. Este método é empregado por Lokalize que permite a definición de scripts para a previsualización de cadeas.

  \item \textbf{Renderizado de interfaces de usuario en XML.} Nas bibliotecas de interfaces modernas existe a posibilidade de definir interfaces en ficheiros XML e despois renderizalas. Neste caso as cadeas a traducir van nestes arquivos e sería posible renderizar esta interface coa tradución que estamos a realizar. Este método aínda que si que amosaría a pantalla onde aparece a cadea actual, so é válido para as cadeas que proveñen destes ficheiros XML e non as definidas da forma tradicional. Un exemplo deste método pódese ver na rede no aplicativo Deckard\footnote{\href{http://deckard.malizor.org/}{deckard.malizor.org}} que permite ver as traducións de aplicativos de GNOME.
\end{itemize}


\subsection{Tradución Directa}
A tradución de cadeas de forma automática empregando algoritmos deseñados a tal proceso e que empregan grandes bases de datos aloxadas, xeralmente en internet. Tanto Google como Microsoft teñen os seus produtos corporativos que fan traducións e existen alternativas libres como OpenTrad ou Apertium. Estas traducións aínda que validas soen ser de baixa calidade polo que necesitan unha revisión.

        \include{020_fundamentos_tecnoloxicos}
        \chapter{Metodoloxía}

Nesta sección descríbese a metodoloxía levada a cabo para a realización deste proxecto. Unha metodoloxía e un conxunto de métodos ou prácticas empregadas para a realización dunha tarefa, neste caso o analise, deseño e implementación dunha nova ferramenta CAT para o proxecto GNOME. Escolleuse unha metodoloxía axil de ciclo incremental. Moitas das prácticas que se tomaron son collidas da metodoloxía eXtreme Programming.

\section{eXtreme Programming}

Foi creada por Kent Beck no ano 1999 e trátase dun dos máis destacados métodos de desenvolvemento áxil. eXtreme Programming (a partir de agora XP) avoga por ciclos de desarrollo moi curtos e elimina os roles clásicos de analista, deseñador e programador. Todo equipo participa en todas as partes do desarrollo. Desta forma Beck define uns valores fundamentais da metodoloxía e unhas prácticas que axudan a adoptar estos valores.

\subsection{Valores de eXtreme Programming}

\subsubsection{Comunicación}
É o primeiro valor de XP. Segundo esta metodoloxía os problemas nos proxectos poden ser traducidos a alguén que non falou con outra persoa sobre algo importante do proxecto. Esta mala comunicación non sucede por casualidade e é debido frecuentemente as malas prácticas. Para solucionar isto, XP inclúe prácticas nas que é necesario a comunicación para levalas a cabo. Ademais a figura do \emph{coach} sirve para mellorar a comunicación daquelas persoas que non o están facendo ben.

\subsubsection{Simplicidade}

A simplicidade non é unha tarefa sinxela. XP fai unha aposta e invita ao programador a pensar no que o proxecto necesita hoxe e non no que vai necesitar nun futuro. Para manter esta simplicidade ao longo do tempo é frecuente a refactorización do mesmo para manter dita simplicidade. A simplificación do deseño e da implementación axiliza tanto o desenvolvemento como o mantemento. A simplicidade require \textbf{comunicación} pois canto máis comunicación teñamos por parte do cliente máis sinxelo simple poderemos facer o sistema e canto máis simple sexa o sistema menos comunicación será necesaria para explicar o sistema.

\subsubsection{Retroalimentación}
A retroalimentación ou feedback é fundamental nesta métodoloxía. Pode actuar en varias escalas de tempo. Obtemos feedback en cuestion de minutos ou días de parte dos test do sistema, das peticións dos clientes ou do director do proxecto. Tamen obtemos feedback o longo dos meses cando o usuario pode analizar as caracteristicas que implementamos. Neste sentido XP aposta por unha posta rápida en produción de forma que teñamos sistemas en produción e en desarrollo de forma paralela. Con isto melloramos o sistema xa que imos obtendo as opinións dos usuarios das decisións que xa tomamos e os erros cometidos non se volven a repetir.

\subsubsection{Coraxe}
O coraxe é unha parte inherente a metodoloxía. É necesario para tirar o traballo de varios días e volver e empezar debido a cambios nos requisitos ou aparición de fallos estructurais. É necesario para ser persistente ca resolución dun problema, as cousas que non se dan resolto un día en horas podense resolver o día seguinte en cuestión de minutos. O coraxe non é útil sen os tres primeiros valores. Con unha boa comunicación existe a posibilidade de facer experimentos con máis risco. A simplicidade permite o programador coñecer mellor o código e polo tanto ser máis valiente a hora de facer cambios. A retroalimentación axuda a que alguén se sinta máis seguro ao facer un cambio.

\subsubsection{Respeto}
Por último é necesario respeto. É necesario que os integrantes do equipo se preocupen polo resto de membros e polo que están facendo. Ademais o equipo debe preocuparse polo propio proxecto. Para que XP funcione os programadores debense sentir parte do proxecto e ter un feedback positivo ao respecto.

\subsection{Prácticas recomendadas por eXtreme Programming}

\subsubsection{O Xogo da Planificación}
A planificación é un dialogo entre a xente do negocio e o equipo técnico. Mentras a parte de negodio decide a importancia dun problema, aprioridade da implementación dunha característica ou outra, a composición das entregas ou as datas das mesmas, o equipo técnico é capaz de estimar canto tempo leva implementar unha característica, ten a capacidade de explicar as consecuencias de certa decisión, sabe como organizarse para levar a cabo unha tarefa e pode facer unha planificación máis detallada.

\subsubsection{Entregas Pequenas}
As entregas ou \emph{releases} deben ser o máis pequenas posibles e conter os requerimentos máis valiosos. Aínda así cada release debe er autocontenida e non ter características implementadas a medias solo para facer o ciclo de entregas máis curto.

\subsubsection{Metáfora}
Cada proxecto feito con XP ten unha metafora. Unha metafora é un simil sinxelo do cl

Esta metafora é util para que os membros do equipo teñan unha visión global do que están facendo. Outras metodoloxías chámanlle a isto \textbf{arquitectura}. O problema con empregar o termino arquitectura é que unha arquitectura non ten necesariamente un sentido de cohesión.

O obxetivo desta metafora é ter unha historia coherente para poder explicarlle o sistema tanto os clientes como aos membros do equipo.

\subsubsection{Deseño simple}
Todos os deseños deben, executar todos os tests, non ter código duplicado, ter o menor número de clases e métodos e todas as partes son importantes para os programadores. Con esta filosofía XP intenta facer un deseño simple para as necesidades actuais do programa. Desta forma a implementación será máis rápida e o tempo de aprendizaxe para os outros membros do equipo será máis curto.

\subsubsection{Testing}
Calquera caracteristica incorporada qeu non inclua un test, simplemente non existe. Os programadores incluen test de unidade para as novas funcionalidades e os clientes crean test funcionales de como esperan que o programa funciona. Ambos test forman parte do código do programa. Non é necesario escribir test para cada metodo pero si para cada método que se expoña.

\subsubsection{Refactorización}
Cando é necesario implementar unha nova caracteristica no programa, os programadores preguntanse se existe unha forma simple de implementala e implementana. Despois analizan o código para ver se existe unha forma de facelo de forma máis simple e que siga executando correctamente todos os tests. Esto chamase refactorizar.

É obvio que traballando desta forma emplease moito máis tempo do necesario para a implementación de cada característica pero desta forma poderemos engadir a seguinte caracteristica nunha cantidade razonable de tempo.

\subsubsection{Programación por parellas}
Todo o código en produción é escrito por parellas de programadores con diferentes roles. Por un lado un dos programadores pensara de forma especifica como implementar un certo método mentras o outro pensará dun xeito máis global e estratéxico. Estas parellas cambian continuamente.

\subsubsection{Pertenza Colectiva}
En XP o código pertence a todo o equipo e se unha persoa ten oportunidade de engadir algo de valor a algún fragmento de código ten que facelo nalgún momento. Desta todo o mundo ten responsabilidade de todo o sistema e aínda que non todo o mundo coñece cada parte de forma igual, todo o mundo coñece algo de cada parte de forma que son capaces de facer modificacións satisfactorias.

Isto contrasta coas prácticas doutras metodoloxías onde o código escrito por unha persoa so pertence a esa persoa e para engadir nova funcionalidade é necerio facer unha petición a dito programador. Esta práctica pode facer máis lento o desarrollo e diminue o factor camión\footnote{\href{http://en.wikipedia.org/wiki/Bus\_factor}{Truck Factor}: O numero de membros dun equipo dentro dun proxecto, que no caso de seren atropellados por un camión, o proxecto non podería completarse.}.

\subsubsection{Integración Continua}
Os test son executados con cada cambio e solo se todos os test son executados correctamente se suben os cambios ao producto final. É importante executar os test a cada cambio xa que asi saberemos a que se debe o fallo e quen ten que correxilo. Se para implementar unha caracteristica os seus desarrolladores non son capaces de que todos os tests funcionen, probablemente necesiten volver a empezar pois non tiñan os coñecementos necesarios para implemementala. 

\subsubsection{Semana de 40 horas}
XP establece unha xornada laboral de 8 horas e 5 días a semana. Para esta metodoloxía é importante que os programadores estean frescos e inspirados cada maña e con xornadas largas de traballo dita tarefa é imposible. O descanso é algo fundamental para poder ter boas idea.

\subsubsection{Cliente no sitio}
Un cliente do proxecto, e dicir, unha persoa que realvente vaia usalo cando esté en produción; debe sentarse xunto o equipo e poder responder preguntas e resolver disputas entre membros do equipo.

Hai casos nos quw esta práctica non é posible debido o alto valor do tempo do cliente. Non obstante temos que pensar nesto como nunha mellora moi substancial da calidade do software final.

\subsubsection{Estandares de programación}
Se traballando cambiando de parellas cada pouco tempo e facemos refactorizacións continuas, isto non pode funcionar sen que todo o equipo siga uns estandares de programación. Isto é un mesmo estico de código e unha mesma forma de facer certas cousas.

\section{Metodoloxía seguida}



        \include{040_planificacion_seguimento}
        \chapter[Análise de Requisitos]{Análise de Requisitos globais}

Neste capítulo explicaremos o proceso de análise de requisitos levado a cabo para a contrución da ferramenta.

\section{Consultas cos traductores}
Para facer unha nova ferramenta de asistencia a tradución consultamos os usuarios principais da ferramenta. Para iso enviamos correos as unhas cantas listas de correo de traductores de diferentes proxectos de software libre como os seguintes:

\paragraph{GNOME} Dentro do proxecto GNOME hai un grupo especifico para a internacionalización dos programas da plataforma. Existen listas de correo para cada linguaxe e unha a nivel internacional. En concreto as listas que enviamos un correo preguntando por ideas para o novo programa foron a lista \emph{internacional}, a lista de traductores o \emph{galego} e a lista de traductores ao \emph{castelán}.

\paragraph{Proxecto Trasno} É unha comunidade de traductores de proxectos de software libre ao Galego. Sirve de punto de encontro para os traductores galegos e realiza periodicamente reunións para a discursión da terminoloxía a usar e a presentación de novas ferramentas.

\paragraph{OpenSuse e Fedora} Estas dúas distribucións de Linux contan cos seus respectivos equipos de traductores. Aínda que a maior parte dos programas destas distribucións son traducidos por outros proxectos, como pasa no caso do entorno de escritorio GNOME, si que hai partes do sistema que necesitan ser traducidas polos propios traductores de cada distribución.

\paragraph{OpenOffice e LibreOffice} Tanto OpenOffice como o seu \emph{fork}, LibreOffice, contan con equipos internacionalización aos que se lles consultou para recoller ideas para o novo programa.

\paragraph{Mozilla} O proxecto Mozilla, autores do navegador Firefox e do xestor de correo electrónico Thunderbird, conta cun equipo de traductores propio. Neste caso non usan ficheiros de tipo Gettext PO, senon ficheiros XML.

Case en todos os correos enviados houbo xente que respondeu dando ideas sobre o que lles gustaría que incorporase o novo programa. Na seguinte sección podemos ver a maioría destas utilidades explicadas.

\subsection{Resumo das peticións}
	\paragraph{Abrir e gardar ficheiros en diferentes formatos.} O programa debería permitir abrir outros formatos de ficheiros aparte do formato Gettext po. Esta petición foi feita sobretodo por membros de equipos de tradutores que non son de GNOME. Existe unha biblioteca de nome \emph{translate-toolkit} que facilita a conversión entre formatos.

	\paragraph{Xestión de cabeceiras} Que o programa modifique automaticamente as cabeceiras con metainformación que tanto o formato Gettext PO como outros tipos de formatos teñen.

	\paragraph{Perfiles de usuarios} Permitir ter diversos perfiles de usuarios para diferentes proxectos de tradución ou para diferentes linguaxes.

	\paragraph{Vista de proxecto} Os traductores consideran moi interesante agrupar os ficheiros relacionados en proxectos e poder ter estatísticas de proxectos.

	\paragraph{Buscar e buscar e reemplazar}  Regex, search in several files/ in the whole document, update translation files with new ones.

	\paragraph{Dividir/Misturar ficheiros} Os traductores consideran que é interesante que en ficheiros grandes exista a posibilidade de dividir e volver a unir ficheiros de tradución para que así sexa posible que máis dunha persona traballe no mesmo ficheiro a vez.

	\paragraph{Medidas Económicas} É interesante incorporar unha ferramenta que permita calcular o custe da tradución efectuada por un traductor. Istó é especialmente util cando falamos de traductores profesionais e non amateur.

	\paragraph{Resaltado da síntaxe} O programa debe resaltar aqueles elementos da cadea que non son traducibles e pertencen o dominio das linguaxes de programación. Por exemplo na seguinte cadea \lstinline[language=C]{"Temos %i coches."} o elemento \lstinline[language=C]{%i} debería estar resaltado xa que so parte do formateado da cadea por parte do programa.

	\paragraph{Comunicación con servidor web} Os traductores consideran a posiblididade de que o porgrama permita certa comuniación con xestores de traducións en internet. No caso de GNOME o programa Damned Lies xestiona os ficheiro .po para todolos programas de GNOME e os traductores deben baixar e subir os ficheiros dende esa plataforma.

	\paragraph{Navegación dentro do documento} A posibilidade de navegar a través das cadeas. Engadir a posibilidade de ir a seguinte cade traducida, sen traducir ou con unha tradución difusa.

	\paragraph{Memoria de tradución} Engadir unha memoria de tradución que permita exportar e importar ficheiros en diferentes formatos. Ter varias memorias  de tradución con prioridade entre elas, engadir a posibilidade de editar a meoria de tradución e acceder a diversos servidores que xestionan memorias de tradución online como poden ser Trobador, amaGama, Open-tran ou Transvision. Import/Export different formats (.po, .tmx, …) Several TM. Priority. Update a TM., TMX Server (Trobador, amaGama, Open-tran and Transvision),

	\paragraph{Glosario} Engadir a posibilidade de consultar a tradución de termos contra ficheiros unha base de datos local ou contra un servidor de glosario como pode ser Terminator.

	\paragraph{Busca de termos} Que o programa permita a busqueda nun dicionario tanto local como en internet. Citanse dicionarios en internet como o que proporciona Wikipedia ou a Universidade de Santiago de Compostela para o galego.

	\paragraph{Previsualización das traducións} Moitas das interfaces que se fan actualmente empregan ferramentas de cuarta xeneración que permitirían xerar a interfaz cas traducións. No caso de GNOME, Glade é o programa que permite a creación de interfaces e xa existe unha ferramenta online que permite a previsualización de traducións de nome Deckard. O que se pide é que o programa incorpore esa posibilidade dende a súa interface.

	\paragraph{Programa controlable totalmente a través do teclado} O programa en xeral peros sobretodo a interface de edición de ficheiros debe ser manexable totalmente a través do teclado para mellorar a productividade dos traductores. Os traductores piden tamén a posibilidade de que se permita personalizar os atallos de teclado.

	\paragraph{Tradución doutras linguaxes} Hai linguas como o galego e o portugués que se parecen moito polo que as veces resulta moi útil poder, en vez de partir de cero, coller unha tradución dun idioma semellante e editala.

	\paragraph{Comprobacións} Os traductores resaltaron a utilidade de que a ferramenta comprobe certa parametros para comprobar a calidade da tradución. Entre outros:
		\begin{itemize}
			\item \textbf{Ortografía e Gramática.} O programa avisará se a cadea traducida ten erros tanto ortográficos e gramaticais.
			\item \textbf{Coherencia terminolóxica.} O programa avisará o usuario cando este empregue unha tradución dun termo diferente a que se ven empregando no resto do ficheiro.
			\item \textbf{Tags XML e marcas de formato.} O programa avisará o usuario se faltan ou están mal escritos os diferentes tags XML ou marcas de formato.
		\end{itemize}

	\paragraph{Multiplataforma} A aplicación debe estar dispoñible para varios sistemas operativos (BSD, Windows, MAC OS, etc.).

	\paragraph{Tradución automatica} O aplicativo debe incorporar mecanismos de tradución automática emptegando ferramentas como Google Translator, Bing Translator, Opentrad ou Apertium.

	\paragraph{Estatísticas} Débese amosar estatísticas tanto a nivel de ficheiro como de proxecto do numero de cadeas ou palabras traducidas, sen traducir ou difusas.

\section{Análise doutros aplicativos do mercado}
Para facer o novo produto software para a tradución tamén consultamos outras ferramentas similares existentes no mercado para intentar imitar as súas vantaxes e evitar as súas deficiencias. O resultado desta análise pódese consultar na Seccion~\ref{sec:ferramentascat}.

\section{Requisitos do aplicativo}
Con toda a información obtida realizamos unha lista de casos de uso que debe cumplir o programa.

\begin{itemize}
  \item \textbf{Abrir ficheiro.} O programa debe ser capaz de abrir ficheiros PO pero tamén ter un deseño extensible que permita abrir outros formatos.
  \item \textbf{Gardar ficheiro.} Debemos poder gardar ficheiros PO pero tamén ter un deseño extensible que permite gardar noutros formatos.
  \item \textbf{Amosar ficheiro.} Debemos amosar o contido dos ficheiros e dicir, as cadeas e as estatísticas deste ficheiro.
  \item \textbf{Editar ficheiro.} Debemos permitir editar o contido dos ficheiros.
  \item \textbf{Buscar cadeas.} A aplicación permitirá buscar entre as cadeas do ficheiro.
  \item \textbf{Navegar polas cadeas.} Permitiremos navegar polas cadeas do ficheiro podendo avanzar entre as cadeas traducidas, sen traducir, etc.
  \item \textbf{Xestionar perfiles.} Permitiremos xestionar diferentes perfiles para os traductores.
  \item \textbf{Obter pistas.} Amosaremos ao usuario pistas que lle indequen que está a facer algo mal.
  \item \textbf{Obter suxerencias.} Debemos amosar ao tradutor diferentes alternativas de traducións.
\end{itemize}

Na figura~\ref{fig:casosdeuso} podemos ver o Diagrama UML de casos de uso para este programa.

\begin{figure}[h!]
    \centering
    \includegraphics[width=0.9\textheight,angle=90]{img/casosdeuso.png}
    \caption{Diagrama UML de casos de uso do aplicativo}
    \label{fig:casosdeuso}
\end{figure}

%
% FIN DEL CAPÍTULO
%

        \chapter{Deseño e Implementación}

\section{Módulo de Ficheiros}

O módulo de xestión de ficheiros é un do máis importantes e complexos de todo o programa o cal é lóxico tendo en conta que o programa é un editor de ficheiros.

En canto ao deseño, o principal obxetivo é a extensibilidade do mesmo pois, aínda que o programa está centrado na edición de ficheiros PO e son ests os unicos soportados na actualidade, pretendese que o programa sexa capaz de soportar varios tipos de ficheiros nun futuro. Na figura~\ref{fig:dia_class:files} pódese ver o diagrama de clases deste módulo.

\subsection{Implementación Xenérica}
A implementación da clase \lstinline{File} contén propiedeades para conseguir información sobre o nome e path onde está dito ficheiro, ademáis garda estatísticas sobre o número de mensaxes traducidos, sen traducir ou con tradución difusa. Estas estatísticas actualizanse cada vez que se engade ou elimina unha cadea ou cada vez que esta se modifica. Tamén contén un valor booleano que permite saber se o ficheiro foi modificado. Aparte de métodos para engadir e eliminar cadeas, e para conseguir e modificar os metadatos do ficheiro está clase ten un método para gardar o ficheiro e para analizar o ficheiro. O método para gardar~(\lstinline{save}) emprega o patrón \emph{Template Method\footnote{\href{http://gl.wikipedia.org/wiki/Template_Method_\%28patr\%C3\%B3n_de_dese\%C3\%B1o\%29}{Template Method (patrón de deseño)}}} o cal permitenos actualizar o estado do ficheiro a non cambiado.

Un ficheiro contén instancias de mensaxes~(\lstinline{Message}). As mensaxes teñen como propiedades un estado, orixes, e consellos. A API provee métodos para conseguir e modificar tanto as cadeas orixinais como as traducións na súa forma en singular ou nalgunha das formas plurais. A implementación do método para modificar unha tradución tamén emprega o patrón Template Method para actualizar o estado da mensaxe. Esta clase tamén ten métodos para engadir e eliminar consellos e para obter o contexto da mensaxe.

\begin{figure}[h!]
    \centering
    \includegraphics[width=\textwidth]{img/genericfile.png}
    \caption{Diagrama de Clases do módulo ficheiros}
    \label{fig:dia_class:files}
\end{figure}

Ademais a clase \lstinline{FileOpener} recolle un método para crear ficheiros e un conxunto de extensións que se poden abrir con este FileOpener.

\subsubsection{Consellos (Tips)}
Os consellos son a solución que damos para aportar información ao usuario sobre a tradución que está a realizar. Algunha das cousas que pode axudar a indicar esta característica é se a tradución está a ser demasiado longa, se hai algunha palabra que está mal escrita na tradución ou se non emprega a terminoloxía adecuada.

Cada consello ten un nome que debe ser xenerico a cada clase de consello, unha descripción que ten que explicar o problema con detalle, un nivel que indica a gravidade do consello e unha referencia a forma plural a que corresponde dito consello. Ademáis pode conter un ou máis referencias a localización exacta do problema que permitirá destacala na interface gráfica.

A creación de consellos faráse ao modificarse unha cadea e farase a través de plugins.

\subsubsection{Pistas (Hints)}
As pistas amosaranlle ao usuario posibles traducións ou aproximacións as traducións. Estas pistas poden ser obtidas de memorias de traducións, de traducións do mesmo ficheiro noutra linguaxe, ou da tradución directa por exemplo.

Cada instancia dunha pista~(\lstinline{Hint}) contén a tradución suxerida, unha cadea que identifica a orixe de dita pista e un valor que indica a precisión de dita suxerencia.

De igual forma que no caso dos consellos a creación de pistas correrá ao cargo de plugins creados a tal proposito. Neste caso actualizaranse as pistas de cada mensaxe ao selecionalo mesmo na interface.

\subsection{Ficheiro PO}
A implementación especifica para ficheiros PO extende as clases \lstinline{File}, \lstinline{FileOpener} e \lstinline{Message} abstractas para implementar os metodos e permitir empregar ficheiros PO.

Para a analise e a actualización dos ficheiros PO empregaremos a biblioteca gettext-po. No momento no que se iniciou a implementación deste módulo non existía implementación destal librería en Vala polo que tivemos que crear uns \emph{bindings} para poder empregala.

Tanto a clase PoFile como a clase PoMessage delegan a maior parte dos seus métodos nas instancias das clases dos bindings da biblioteca GettextPo File e Message respectivamente. Desta forma faise un claro uso do patrón \emph{Adapter\footnote{\href{http://gl.wikipedia.org/wiki/Adapter_\%28patr\%C3\%B3n_de_dese\%C3\%B1o\%29}{Adapter (patrón de deseño)}}}.

\begin{figure}[h!]
    \centering
    \includegraphics[width=\textwidth]{img/pofile.png}
    \caption{Diagrama de Clases do ficheiro PO}
    \label{fig:dia_class:pofile}
\end{figure}

De forma adicional a clase PoFile contén unha instancia da clase PoHeader esta clase que extende e PoMessage correspondese aos metadatos que se atopan nos ficheiros PO na tradución da cadea baleira. É ten meodos para obter e modificar metadatos e para actualizar os datos dos autores das traducións de dito ficheiro. A clase PoFile delega nesta clase a hora de conseguir e modificar metadatos e tamén cando garda un ficheiro.

A clase \lstinline{PoFileOpener} so é capaz de abrir ficheiros con extensión PO e simplemente emprega o método parse da clase ficheiro para crear unha nova instancia.

\subsubsection{Implementación dos bindings}
Como xa mencionamos vimonos na obriga de crear nós os bindings para a biblioteca gettext-po. Vala é unha linguaxe deseñada para permitir o acceso a outras bibliotecas escritas en C, especialmente se se trata de bibliotecas baseadas en GObject. Despois de todo Vala emprega C como linguaxe intermedio. A biblioteca gettextpo non é unha biblioteca baseada en GObject polo que a creación destes bindings é un pouco máis complicado.

Para poder empregar unha biblioteca escrita en C no noso programa escrito en Vala so temos que crear un ficheiro con extensión VAPI que conteña en sintaxe Vala as clases da biblioteca engadindo unhas etiquetas que permitan a súa tradución ao código C correcto. No Fragmento de código~\ref{lst:bindingsgettext} podemos ver parte dos bindings creados para a biblioteca gettextpo.

\lstset{language=[sharp]C}
\begin{lstlisting}[label=lst:bindingsgettext,caption=Bindings da biblioteca GettextPo]
[CCode (cprefix = "Po", lower_case_cprefix = "po_")]
namespace GettextPo {

    [CCode(cheader_filename = "gettext-po.h", cname="struct po_file", free_function="po_file_free")]
    [Compact]
    public class File {

        [CCode (CCode = "po_file_create")]
        public File();

        [CCode (array_length = false, array_null_terminated = true, cname="po_file_domains")]
        public unowned string[] domains ();

        [CCode (cname="po_file_write")]
        public static unowned GettextPo.File file_write (GettextPo.File file,
                            string filename,
                            XErrorHandler handler);
        [...]
    }

    [CCode(cheader_filename = "gettext-po.h", cname="struct po_message")]
    [Compact]
    public class Message  {

        [CCode (cname="po_message_create")]
        public Message();
        public unowned string msgid ();
        public unowned string? msgid_plural ();
        [...]
\end{lstlisting}

Como podemos ver o traballo prácticamente limitase a establecer o parametro \lstinline{cname} en cada clase e método. Hai que ter en conta que non sempre e necesario especificar este parámetro pois as bibliotecas empregan case sempre unhas normas de nombrado que fan que usen primeiro o nome da biblioteca, despois o nome da clase e logo o nome do método separado por barras baixas. Desta forma no método da biblioteca \lstinline{po_message_msgid()}, \emph{po\_} corresponde o nome da biblioteca, \emph{message\_} ao nome da clase e \emph{msgid} ao nome do método. Os bingings de vala empregan estas normas para nomear os métodos polo que en ocasións non é necesario especificar o parametro cname. Isto sucede, como se pode ver no fragmenteo de código anterior no caso da clase Mensaxe (\lstinline{Message}).

Ademais e necesario especificar o nome do ficheiro cabeceira que se emprega e no caso de que o tipo de retorno sexa un array temos que especificar máis parametros como acontece no método \lstinline{domains()} da clase File. Por último é importa especificar a pertenza de cada valor e os métodos correctos para liberar as instancias desta forma evitaremos que haxa perdas de memoria e fallos de segmentación.

\section{Módulo de Linguaxes}
O módulo de linguaxes xestiona os linguaxes e formas plurais existentes. Contén dúas clases, a clase \lstinline{Language} e a clase \lstinline{PluralForm}. Estas dúas clasen proveen un método estático para obter as instancias existentes. Estas instancias creanse a primeira vez que se carga a clase consultando unhas bases de datos que consisten nuns ficheiros JSON\footnote{\href{http://gl.wikipedia.org/wiki/JSON}{JSON}}.

A cada instancia dunha linguaxe contén o nome da linguaxe, o seu código empregando a norma ISO 639-1 a súa forma plural e o email do equipo de tradución por defecto. Engadimos o idioma do equipo de tradución por defecto para autocompletar este campo no perfil do usuario. No Fragmento de Código~\ref{lst:languages} podemos ver un fragmento do ficheiro JSON contendo as linguaxes.

\begin{lstlisting}[label=lst:languages,caption=Fragmento da Base de Datos de Linguaxes]
  {
    "languages" : [
      [...]
      {
        "code" : "sl",
        "name" : "Slovenian",
        "pluralform" : "nplurals=4; plural=(n\%100==1 ? 1 : n\%100==2 ? 2 : n\%100==3 or n\%100==4 ? 3 : 0);",
        "default-team-email": ""
      },
      {
        "code" : "so",
        "name" : "Somali",
        "pluralform" : "",
        "default-team-email": ""
      },
      {
        "code" : "es",
        "name" : "Spanish; Castilian",
        "pluralform" : "nplurals=2; plural=(n != 1);",
        "default-team-email": "gnome-es-list@gnome.org"
      },
      [...]
    ]
  }
\end{lstlisting}

En canto a clase \lstinline{PluralForm}, esta clase inclue o número de plurais, a expresión desa forma plural e un conxunto de tags. Estos tags pretenden faculitar ao usuario a identificación de que número corresponde a cada forma plural. No Fragmento de Código~\ref{lst:plurals} pódemos ver unha parte da base de datos de formas plurais.

\begin{lstlisting}[label=lst:plurals,caption=Fragmento da Base de Datos de Plurais]
  {
    "forms" : [
      {
        "expression" : "nplurals=2; plural=(n > 1);",
        "number_of_plurals" : 2,
        "tags" : [
	  {
            "number" : 0,
            "tag" : "Equal to 0 or 1"
	  },
	  {
            "number" : 1,
            "tag" : "Greater than 1"
	  }
	]
      },
      [...]
    ]
  }
\end{lstlisting}

\section{Interface Gráfica}
A interface gráfica é unha das partes a que máis tempo lle dedicamos neste proxecto. O obxetivo dende o principio foi construir algo simple pero potente que permitira ao usuario editar os ficheiros PO de forma sinxela pero que aportase fose capaz de aportarlle moita información que lle axudase a facer a tradución.

\subsection{Evolución}
Durante a execución do proxecto probamos diferentes formas da interface gráfica ata chegar ao resultado actual. Estas versións pretenden imitar programas existenttes e obedecen xeralmente a conversacións con traductores.

\subsubsection{Primeira Versión: moi semellante a GTranslator}
Para comezar a traballar e tras amosar varios mockups nos reportes recibindo feedback sobre eles decidimos facer unha interface de usuario moi parecida a de GTranslator. Esta interface inclue bloques para a lista de mensaxes, editar ditos mensaxes e amosar o contexto. Estos bloques podense mover por toda a interface e incluso separar da mesma xa que estamos empregando a biblioteca GNOME Docking Library. Na Figura~\ref{fig:ui:v1:general} podemos ver o aspecto desta primeira versión da interface.

\begin{figure}[h!]
  \centering
    \includegraphics[width=\textwidth]{img/gsoc1_it2_ui.png}
    \caption{Diagrama de Clases do módulo ficheiros}
    \label{fig:ui:v1:general}
\end{figure}

Como podemos ver a interface contén un barra de ferramentas que permite abrir ficheiros, gardalos, desfacer e refacer cambios, buscar no documento e ver as preferencias. Permite abrir diversos ficheiros en varias lapelas. En cada unha das lapelas podemos ver a lista de mensaxes e o widget de edición.

En cada mensaxe da lista, aparte das cadeas orixinal e traducida, podemos ver o estado da cadea e se está ten consellos (Tips) activos e de que nivel de estos. Por outro lado, no widget de edición podemos ver unha lapela por cada forma plural que podemos editar e unha lista vertical con iconos cos consellos. Ao pasar o rato polo consello veremos a súa descripción.

\begin{figure}[h!]
  \centering
  \includegraphics[width=0.4\textwidth]{img/gsoc1_it3_ui.png}
  \includegraphics[width=0.4\textwidth]{img/gsoc1_it5_prefs.png}
  \caption{Diagrama de Clases do módulo ficheiros}
  \label{fig:ui:v1:dialogs}
\end{figure}

Como podemos ver na Figura~\ref{fig:ui:v1:dialogs} tanto a hora de xerar unha búsqueda como para ver as preferencias esta interface empregara uns dialogos non modales.

En conversacións en IRC con traductores e anteriores mantenedores de GTranslator chegamos a conclusión de que o uso de GNOME Docking Library era unha mala idea pois significaba un erro de deseño xa que se o usuario tiña que modificaar o apecto da interface é que o deseño non era correcto. Esta biblioteca ademais ten bastantes fallos polo que non era recomendable usala.

Ademais durante a GUADEC, comentaronos da existencia dun widget moito mellor para xestionar a buscas consistente nunha barra horizontal que se desplegaba cando a busca estaba activa.

Por outro lado o resultado desta modelo de interface tampouco era demasiado satisfactorio polo que intentamos probar con unha versión máis cercana ao programa Virtaal.

\subsubsection{Segunda Versión: semellante a Virtaal}
\subsubsection{Terceira Versión}

\subsection{Paneis}

\begin{figure}[h!]
  \centering
    \includegraphics[width=0.8\textwidth]{img/panel_abrir_ficheiro.png}
    \caption{Diagrama de Clases do módulo ficheiros}
    \label{fig:dia_class:files}
\end{figure}

\begin{figure}[h!]
    \centering
    \includegraphics[width=0.8\textwidth]{img/panel_edicion.png}
    \caption{Diagrama de Clases do módulo ficheiros}
    \label{fig:dia_class:files}
\end{figure}

\begin{figure}[h!]
    \centering
    \includegraphics[width=0.8\textwidth]{img/panel_ficheiros_abertos.png}
    \caption{Diagrama de Clases do módulo ficheiros}
    \label{fig:dia_class:files}
\end{figure}

\begin{figure}[h!]
    \centering
    \includegraphics[width=0.8\textwidth]{img/panel_pefil_xeral.png}
    \caption{Diagrama de Clases do módulo ficheiros}
    \label{fig:dia_class:files}
\end{figure}

\begin{figure}[h!]
    \centering
    \includegraphics[width=0.8\textwidth]{img/panel_perfil_avanzado.png}
    \caption{Diagrama de Clases do módulo ficheiros}
    \label{fig:dia_class:files}
\end{figure}

\begin{figure}[h!]
    \centering
    \includegraphics[width=0.8\textwidth]{img/panel_preferencias_edicion.png}
    \caption{Diagrama de Clases do módulo ficheiros}
    \label{fig:dia_class:files}
\end{figure}

\begin{figure}[h!]
    \centering
    \includegraphics[width=0.8\textwidth]{img/panel_preferencias_perfiles.png}
    \caption{Diagrama de Clases do módulo ficheiros}
    \label{fig:dia_class:files}
\end{figure}

\begin{figure}[h!]
    \centering
    \includegraphics[width=0.8\textwidth]{img/panel_preferencias_plugins.png}
    \caption{Diagrama de Clases do módulo ficheiros}
    \label{fig:dia_class:files}
\end{figure}


\section{Navegación e Busca a través do documento}

\section{Preferencias}

\subsection{Módulo de Perfiles}

\section{Plugins}

        \include{070_conclusions}

  % INCLUIMOS LOS APÉNDICES...
        \appendix
        \chapter{Escoller a licencia do programa}

A hora de escoller unha licencia para o novo programa temos que ter en conta as licencias das bibliotecas empregadas no noso programa. Na siguente lista podense ver as bibliotecas empregadas para o programa así como a licencia de cada unha delas.

\begin{itemize}
  \item \textbf{GLib} GNU Lesser General Public License v2.1
  \item \textbf{GTK+} GNU Lesser General Public License v2.1
  \item \textbf{LibGee}  GNU Lesser General Public License v2.1
  \item \textbf{LibPeas} GNU Lesser General Public License v2.1
  \item \textbf{GettextPo} GNU General Public License v2
  \item \textbf{GTKSourceView} GNU Lesser General Public License v2.1
  \item \textbf{JSON-GLib} GNU Lesser General Public License v2.1
\end{itemize}

Como podemos ver excepto a biblioteca GettextPO, que emprega unha licencia GNU GPLv2, o resto de bibliotecas empregadas usan unha licencia GNU LGPL v2.1. A licencia GNU Lesser GPL permite empregar calquera licencia en productos derivados. Por outra parte a licencia GNU GPL v.2 obliga a empregar a licencia GNU GPL v2 ou versións posteriores da mesma licencia.

No noso caso eleximos a licencia GNU General Public Licence v3 que é unha versión máis actualizada da licencia para programas de GNU. Na seguinte sección podemos ver o texto completo desta licencia.

        \include{A011_textogpl}
       %\include{pfc_appendix_020}


  % INCLUIMOS LA BIBLIOGRAFÍA...
        \nocite{*}  % Se usa para indicar en la bibliograf? las referencias no citadas.
        \bibliography{pfc_biblio}
        \bibliographystyle{plain}

\end{document}

