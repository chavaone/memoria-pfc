\chapter{Fundamentos Tecnolóxicos}

Neste capítulo detallaremos os compoñentes tecnolóxicos que empregamos para levar a cabo este proxecto. Primeiro falaremos das ferramentas empregadas e despois das diferentes bibliotecas que usamos no programa.

\section{Ferramentas empregadas}

\paragraph{Vala} É unha linguaxe de programación orientada a obxectos creada por GNOME para acercar a linguaxe de programación C as características das linguaxes de programación modernas. Emprega C xunto coa biblioteca GObject como linguaxe intermedio. Esta linguaxe cunha sintaxe moi parecida a C\# incorpora características como o uso e funcións anónimas, bucles foreach, xestión automática de memoria ou o manexo de excepcións.


 \paragraph{Autotools} Tamén coñecidas como \emph{GNU build system} son un conxunto de ferramentas que permiten permitir a configuración e compilación de programas portables a varias plataformas UNIX.

 \paragraph{Git} É un sistema de control de versións distribuído escrito por Linus Torbards e usado, entre outros moitos proxectos, para xestionar o desenvolvemento de Linux (o kernel). Trátase dunha das solucións de control de versións máis empregada na actualidade pois é moito máis rápida que as outras existentes e, ao ser distribuída, permite traballar ser conexión a internet.

 \paragraph{GitHub} Unha plataforma de almacenamento de proxectos empregando o sistema de control de versións Git. Permite subir proxectos Open Source de forma gratuíta os seus servidores. Conta ademais con ferramentas para a creación de Wikis relacionadas cos proxectos e incorpora un xestor de fallos.

 \paragraph{Listas de Correo} As listas de correo son un medio de comunicación asíncrona moi usadas no mundo do desenvolvemento software. Os usuarios escriben un correo electrónico a un enderezo especial que reenvia dito correo a todos os usuarios subscritos as listas. Ademais os correos enviados a lista son almacenados nun historial para a súa posterior consulta.

 \paragraph{IRC (Internet Relay Chat)} É un protocolo de comunicación entre usuarios creado no ano 1988. É amplamente usado no mundo do desenrolo software como unha ferramenta rápida para consultar dubidas entre desenvolvedores. Existen múltiples clientes dispoñibles para case calquera plataforma. En concreto empregouse o programa XChat.

 \paragraph{JHBuild} Unha serie de scripts que permiten a descarga automática do código fonte dun proxecto e das súas dependencias e a instalación destes proxectos dentro dun entorno separado do resto do sistema. Desta forma os desenvolvedores poden traballar coas últimas versións das bibliotecas que tenden a ser inestables sen ter que usalas en todo os seu sistema.

\paragraph{Glade} Unha ferramenta para a creación de interfaces de usuario con GTK+ as interfaces creadas expórtanse en ficheiros con formato XML.

\paragraph{Dia} É un programa para a creación de diagramas de todo tipo. Ten un paquete para facer diagramas UML.

\paragraph{JSON} É un formato para o intercambio de datos. Acrónimo de JavaScript Object Notation, trátase dun subconxunto da notación para definir obxectos en JavaScript. Para poder analizar este formato no noso programa empregaremos a biblioteca JSON-GLib.

\paragraph{GDB} GNU Project debugger. É o depurador estándar para o proxecto GNU. Permite a execución paso a paso, monitorizar e alterar o valor das variables entre outras moitas características.

\paragraph{\LaTeX} É un sistema de composición de textos. Creado por Leslie Lamport como un gran conxunto de macros de \TeX para facilitar o seu uso, produce documentos de gran calidade con pouco esforzo. Permite incluír dentro dos textos formulas matemáticas, fragmentos de códigos ou figuras, entre outros.

\section{Bibliotecas empregadas}

\paragraph{GLib} É unha biblioteca de propósito xeral creada por GNOME a partir de 5 librerías, GObject, GLib, GModule, GThread e GIO.

\emph{GObject} incorpora características da orientación a obxectos a linguaxe de programación C, como pode ser a creación de clases, herdanza ou as propiedades entre outras. \emph{GLib} constitúe un conxunto de tipos básicos empregados no resto de bibliotecas. \emph{GModule} permite a carga de módulos ou extensións de forma dinámica. Por último, \emph{GThread} é unha biblioteca de xestión de fíos de execución e \emph{GIO} é unha biblioteca pensada para facilitar a entrada e saída nos programas.

Esta biblioteca constitúe unha dependencia básica en todas os aplicativos feitos co \emph{stack} de GNOME e, de feito, é unha dependencia de case todas as demais bibliotecas que empregamos.

\paragraph{GTK+} Biblioteca multiplataforma para a creación de interfaces gráficas de usuario desenvolvida por GNOME. Incorpora unha serie de widgets para a creación de programas tanto grandes como pequenos. Está escrita en C pero existen bindings para moitas linguaxes.

As interfaces gráficas pódense crear de forma programática ou cargando un ficheiro XML. Aínda que pode parecer moito máis traballoso a creación dun ficheiro XML con ese propósito, a existencia de ferramentas como Glade\footnote{\href{http://glade.gnome.org}{glade.gnome.org}}.

\paragraph{LibGee} Trátase dunha biblioteca que fornece a GObject de estruturas de datos comunmente usadas como listas, táboas hash ou conxuntos entre outros.

\paragraph{LibPeas} LibPeas é un motor de plugins para GObject. Escrito orixinalmente para GEdit permite as aplicacións estenderse a través dun sistema de plugins.

\paragraph{GettextPo} Trátase dunha biblioteca que analiza ficheiros GNU Gettext PO. Esta biblioteca está implementada en C e non existen bindings para Vala polo que teremos que facelos nós.

\paragraph{GTKSourceView} Esta biblioteca estende o \emph{widget} de GTK+ GtkTextView, que permite a visualización de textos, para incorporarlle certas características avanzadas como a xestión de cambios permitindo desfacer e refacer ou o resaltado da sintaxe.

\paragraph{JSON-GLib} É unha biblioteca para a análise de cadeas de texto en formato JSON.
