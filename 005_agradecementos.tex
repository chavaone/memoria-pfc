%
% Agradecimientos
%

\section*{Agradecementos}

Quero aproveitar este espazo para amosar o meu agradecemento a todas aquelas persoas que dunha maneira ou outra estiveron relacionadas coa elaboración deste proxecto. Gustaríame agradecer a axuda prestada polos mentores dos dous Summer of Code que fixen e a guía que me ofreceu Fernando Bellas para poder presentar a realización deste programa como o meu Proxecto de Fin de Carreira. Quero, ademais, facer facer unha especial mención a Daniel Mustieles, coordinador do equipo de tradutores de GNOME ao castelán, pai deste proxecto e a persoa que máis me axudou para realizalo. Tamén, expresar o meu agradecemento a todos os usuarios e desenvolvedores de GNOME que dende un chat de IRC, un correo ou un comentario dun blogue me axudaron a mellorar o programa.

Gustaríame agradecer tamén as horas de dedicación de todo o persoal docente que dende os tres anos cos que empecei a ir a escola, ata as últimas materias da carreira me estiveron formando. Sen todo o tempo que dedicaron a formarme non tería os coñecementos necesarios sobre a informática e sobre a vida para poder levar a cabo o presente proxecto.

Non me quero esquecer de todas o resto de persoas que me acompañan neste camiño que é a vida. Quero dar grazas a meus pais, tíos, avós  e o resto da miña familia que sempre me apoiou canto puido e máis.

Teño a sorte de contar con moitos e bos amigos e amigas así que por último e non por elo menos importante gustaríame darlles a todos eles as grazas por eses anacos de tempo agora xa inmortais nos que xunto a eles tiven a oportunidade de viaxar, saltar, bailar, estudar, rir, chorar, falar, cantar, berrar, comer, soñar, beber... en definitiva, de vivir. 

A todos vós, moitas grazas!


\begin{flushright}
  Marcos Chavarría Teijeiro \\
  A Coruña, 15 de xuño de 2015
\end{flushright}

